\documentclass[palatino,nochap]{apuntes}

\title{Frecuencias obtenidas}
\author{Daniel Ruiz Mayo, Alberto Javier Parramón, Víctor de Juan}
\date{}

% Paquetes adicionales

% --------------------

\begin{document}
\pagestyle{plain}
\maketitle

Las graficas corresponden a los datos de frecuencias, número de documentos en los que aparece el término, tf promedio e idf del término.
Se ha empleado la escala logaritmica para la representación de los datos correspondientes a frecuencias y número de documentos para su mejor visualización.

Tando el tf como el idf utilizados han sido los vistos en clase.

\[ tf(t,d) = \left\{ \begin{array}{cc} 1+\log_2frec(t,d) & \text{ si } frec(t,d) > 0\\ 0 & \text{ en otro caso}\end{array}\right. \]

\[ idf(t) = \log \frac{|D|}{|D_t|} \]

A continuación, incluimos las gráficas obtenidas:


\begin{center}\includegraphics{img/frecuencias10k.png}
\end{center}
\begin{center}\includegraphics{img/frecuencias1k.png}
\end{center}
\begin{center}\includegraphics{img/idf10k.png}
\end{center}
\begin{center}\includegraphics{img/idf1k.png}
\end{center}
\begin{center}\includegraphics{img/numdocs10k.png}
\end{center}
\begin{center}\includegraphics{img/numdocs1k.png}
\end{center}
\begin{center}\includegraphics{img/tf10k.png}
\end{center}
\begin{center}\includegraphics{img/tf1k.png}
\end{center}

% Contenido.

%% Apéndices (ejercicios, exámenes)
\appendix

\end{document}
