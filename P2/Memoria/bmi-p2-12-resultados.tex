\documentclass[palatino,nochap]{apuntes}

\usepackage{hyperref}
\usepackage{vmargin}

\def\changemargin#1#2{\list{}{\rightmargin#2\leftmargin#1}\item[]}\let\endchangemargin=\endlist

\title{Resultados práctica 2}
\author{Daniel Ruiz Mayo, Alberto Javier Parramón, Víctor de Juan}
\date{}



% Paquetes adicionales

% --------------------

\begin{document}
\pagestyle{plain}
\maketitle

% Contenido.

%% Apéndices (ejercicios, exámenes)
%\appendix

\section{Creación del índice}
Para la creación del índice hemos utilizado una máquina con las siguientes características:

\begin{itemize}
	\item \textbf{RAM: }5,7 GB (Usamos hasta 5 GB en la máquina virtual de Java)
	\item \textbf{Procesador: } Intel® Core™ i7-4702MQ CPU @ 2.20GHz × 8 
\end{itemize}

Además, como ya hemos indicado, leemos los documentos uno a uno del zip hasta leer un total de 400 MB de datos. Es en ese momento cuando fusionamos el índice almacenado en RAM con el que tenemos en disco.

El resumen de los tiempos y el uso de memoria del indexado es el siguiente:



\begin{tabular}{|c|c|c|c|}
	\hline
	 & Tiempo de & Consumo máximo & Tamaño total \\
	 & indexado & de RAM en indexado & del índice en disco  \\
	\hline
	1K & 4min 36s & 1028 MB  & 43 MB \\
	\hline
	10K & 40min 37s & 2100 MB & 107 MB \\
	\hline
	100K & 36horas 25min &  4700 MB  & 310 MB\\
	\hline
\end{tabular}


\newpage
\section{Resultados búsqueda}
Para las búsquedas hemos obtenidos los siguientes resultados de evaluación:

Para los Indices de 1k, los P@5 son los siguientes:
\begin{itemize}
\item Para el indice literal basico: $0$.
\item Para el indice literal stemming $0$.
\item Para el indice literal stop-words $0$.
\item Para el indice basico $0.12$.
\item Para el indice stemming $0.12$.
\item Para el indice stop-words $0.12$.
\end{itemize}


Para los Indices de 10k, los P@5 son los siguientes:
\begin{itemize}
	\item Para el indice literal basico: $0.08$.
	\item Para el indice literal stemming $0$.
	\item Para el indice literal stop-words $0.08$.
	\item Para el indice basico $0$.
	\item Para el indice stemming $0$.
	\item Para el indice stop-words $0$.
\end{itemize}


Para los Indices de 10k, los P@5 son los siguientes:
\begin{itemize}
	\item Para el indice literal basico: $0.04$.
	\item Para el indice literal stemming $0$.
	\item Para el indice literal stop-words $0.04$.
	\item Para el indice basico $0.04$.
	\item Para el indice stemming $0$.
	\item Para el indice stop-words $0.04$.
\end{itemize}


Como ejemplos de los resultados de las querys, se puede ver cualquiera de ellas en los ficheros del directorio $salidas\_queries$, cuyos nombres son los suficientemente descriptivos para saber sobre que índice se ha buscado.


\end{document}
