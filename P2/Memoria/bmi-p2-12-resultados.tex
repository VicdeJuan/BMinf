\documentclass[palatino,nochap]{apuntes}

\usepackage{hyperref}
\usepackage{vmargin}

\def\changemargin#1#2{\list{}{\rightmargin#2\leftmargin#1}\item[]}\let\endchangemargin=\endlist

\title{Resultados práctica 2}
\author{Daniel Ruiz Mayo, Alberto Javier Parramón, Víctor de Juan}
\date{}



% Paquetes adicionales

% --------------------

\begin{document}
\pagestyle{plain}
\maketitle

% Contenido.

%% Apéndices (ejercicios, exámenes)
%\appendix

Para la creación del índice hemos utilizado una máquina con las siguientes características:

\begin{itemize}
	\item \textbf{RAM: }5,7 GB (Usamos hasta 5 GB en la máquina virtual de Java)
	\item \textbf{Procesador: } Intel® Core™ i7-4702MQ CPU @ 2.20GHz × 8 
\end{itemize}

Además, como ya hemos indicado, leemos los documentos uno a uno del zip hasta leer un total de 400 MB de datos. Es en ese momento cuando fusionamos el índice almacenado en RAM con el que tenemos en disco.

El resumen de los tiempos y el uso de memoria tanto del indexado como de la búsqueda los mostramos en la siguiente tabla:

\small{
\begin{changemargin}{-2cm}{0cm}
\begin{tabular}{|c|c|c|c|c|c|}
	\hline
	 & Tiempo de & Consumo máximo & Tamaño total & Tamaño total de respuesta & Consumo máximo  \\
	 & indexado & de RAM en indexado & del índice en disco & a la batería de consultas & de RAM en consulta \\
	\hline
	1K & 4:36 & 1028 MB  & 43 MB & & \\
	\hline
	10K & 40:37 & 2100 MB & 107 MB & & \\
	\hline
	100K &  &  & & & \\
	\hline
\end{tabular}
\end{changemargin}
}

\end{document}
