\documentclass[palatino,nochap]{apuntes}

\title{Memoria práctica 2}
\author{Daniel Ruiz Mayo, Alberto Javier Parramón, Víctor de Juan}
\date{}

% Paquetes adicionales

% --------------------

\begin{document}
\pagestyle{plain}
\maketitle

% Contenido.

%% Apéndices (ejercicios, exámenes)
%\appendix


\section{Creación del índice}

Para la creación del índice hemos utilizado una máquina con las siguientes características:

\begin{itemize}
	\item \textbf{RAM: }5,7 GB (Usamos hasta 5 GB en la máquina virtual de Java)
	\item \textbf{Procesador: } Intel® Core™ i7-4702MQ CPU @ 2.20GHz × 8 
\end{itemize}

EL método $build()$ de la clase $BasicIndex$, crea un índice a partir de un conjunto de documentos comprimidos en un zip.

Para ello, carga en RAM el índice relativo a esos documentos documentos hasta un máximo de 400 MB, cuándo llena ese espacio, lo fusiona con el índice que se encuentra guardado en disco. Este índice sigue la siguiente estructura:

\begin{verbatim}
linea: termino ESPACIO lista_de_postings

lista_de_postings: posting
	| posting ESPACIO lista_de_postings

posting: docId COMA num_posiciones COMA posiciones

posiciones: long
	| long COMA posiciones

\end{verbatim}

Por ejemplo:

\begin{verbatim}
libro 1,3,1,2,3 2,2,1,2
\end{verbatim}

Significa que el término $libro$ aparece en el documento con Id $1$, 3 veces, en las posiciones 1,2 y 3. Y en el documento con Id $2$, dos veces, en las posiciones 1 y 2.

Obtenemos los siguientes resultados de rendimiento:

\begin{tabular}{|c|c|c|c|}
	\hline
	1K & 10K & 100K &  \\
	\hline
	0 & 1 & 1 & 1 \\
	\hline
	1 & 1 & 1 & 1 \\
	\hline
\end{tabular}

\section{Búsqueda}

\end{document}
